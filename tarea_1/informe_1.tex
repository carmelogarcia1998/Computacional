\documentclass[12pt]{article}
 
\usepackage[margin=1in]{geometry} 
\usepackage{amsmath,amsthm,amssymb}
\usepackage[spanish]{babel}
\usepackage[utf8]{inputenc}
\usepackage[T1]{fontenc}

\usepackage{listings}
\usepackage{xcolor}

\definecolor{codegreen}{rgb}{0,0.6,0}
\definecolor{codegray}{rgb}{0.5,0.5,0.5}
\definecolor{codepurple}{rgb}{0.58,0,0.82}
\definecolor{backcolour}{rgb}{0.95,0.95,0.92}

\lstdefinestyle{mystyle}{
	backgroundcolor=\color{backcolour},   
	commentstyle=\color{codegreen},
	keywordstyle=\color{magenta},
	numberstyle=\tiny\color{codegray},
	stringstyle=\color{codepurple},
	basicstyle=\ttfamily\footnotesize,
	breakatwhitespace=false,         
	breaklines=true,                 
	captionpos=b,                    
	keepspaces=true,                 
	numbers=left,                    
	numbersep=5pt,                  
	showspaces=false,                
	showstringspaces=false,
	showtabs=false,                  
	tabsize=2
}
 
\newcommand{\N}{\mathbb{N}}
\newcommand{\Z}{\mathbb{Z}}
 
\newenvironment{theorem}[2][Theorem]{\begin{trivlist}
\item[\hskip \labelsep {\bfseries #1}\hskip \labelsep {\bfseries #2.}]}{\end{trivlist}}
\newenvironment{lemma}[2][Lemma]{\begin{trivlist}
\item[\hskip \labelsep {\bfseries #1}\hskip \labelsep {\bfseries #2.}]}{\end{trivlist}}
\newenvironment{exercise}[2][Exercise]{\begin{trivlist}
\item[\hskip \labelsep {\bfseries #1}\hskip \labelsep {\bfseries #2.}]}{\end{trivlist}}
\newenvironment{problem}[2][Problem]{\begin{trivlist}
\item[\hskip \labelsep {\bfseries #1}\hskip \labelsep {\bfseries #2.}]}{\end{trivlist}}
\newenvironment{question}[2][Question]{\begin{trivlist}
\item[\hskip \labelsep {\bfseries #1}\hskip \labelsep {\bfseries #2.}]}{\end{trivlist}}
\newenvironment{corollary}[2][Corollary]{\begin{trivlist}
\item[\hskip \labelsep {\bfseries #1}\hskip \labelsep {\bfseries #2.}]}{\end{trivlist}}

\newenvironment{solution}{\begin{proof}[Solution]}{\end{proof}}
 
\begin{document}

\title{Tarea 1}
\author{Carmelo García\\ 
Física Computacional}

\maketitle

\section{Problema}

Considere un conjunto de \( N \) partículas que reposan en las posiciones \(\vec{r}_n = x_n \hat{x} + y_n \hat{y}\), tal que:

\[
x_n = -2.7 x_{n-1} \ln(x_{n-1}), \quad n = 1, 2, \ldots, N - 1; \quad x_0 = 0.3333
\]

\[
y_n = 0.9 \sin(\pi y_{n-1}), \quad n = 1, 2, \ldots, N - 1; \quad y_0 = 0.7453
\]

La masa de la n-ésima partícula está dada por \( m_n = |x_n y_n|^{1.3} e^{-r_n / \sqrt{2}} \). Para el caso \( N = 10^8 \), elabore un programa que evalúe para este sistema:

\begin{itemize}
    \item[a)] El área del mínimo rectángulo, cuyos lados son paralelos a los ejes \( x \) e \( y \), que encierran este conjunto de partículas.
    \item[b)] La masa total.
    \item[c)] El centro de masas.
    \item[d)] El momento de inercia alrededor del eje que descansa en el plano \( xy \), es paralelo a la recta \( y = x \) y pasa por el centro de masas.
\end{itemize}

\section{Análisis del problema}\label{analisis}

Por ahora no he dimensionando el verdadero poder de \verb|C|, no se si $N = 10^{8}$ partículas es bastante para este lenguaje o es un pequeño tramite.  
No me he querido arriesgar y es tratado de guardar lo menos posible en la memoria. Los cálculos se han hecho sobre la marcha y sòlo se ha ¨guardado" lo necesario. 

La idea principal de la resolucion del problema se basa en tener un solo bucle \verb|for| que iterará sobre todas las partículas, es decir, desde la partícula $0$ hasta la partícula $N$.
Cuando el bucle \verb|for| se centre en cada partícula se van a hacer todos los calculos necesarios para esa partícula. 

A continuación nos vamos a detener en las constantes que vamos a utilizar a lo largo del código:

\begin{lstlisting}[language=C, caption={Declaraciòn de Variables}, style=mystyle]
    #include<stdio.h>
    #include<stdlib.h>
    #include<math.h>
    
    // Definimos las constantes que vamos a usar.
    int N = 100000000;
    float x_0 = 0.3333, y_0 = 0.7453;
    double x_n, y_n, m_n, r_n, x_ant, y_ant, m_t, x_cm=0, y_cm=0, r_cm[2],
           I=0, x_max, x_min, y_max, y_min, A;
    \end{lstlisting}

\begin{itemize}
    \item $N$: Número de particulas en el sistema.
    \item $x_0 y y_0$: Posición de la partícula inicial. 
    \item $x_n y y_n$: Son las componentes de del vector posición.
    \item $x_ant y y_ant$: Harán el papel de $x_{n-1} y y_{n-1}$.
    \item $r_n$: Es el módulo del vector $\vec{r}_n$, es decir, $\sqrt{ x^2_n + y^2_n}$
    \item $x_cm y y_cm$: Son las componentes de del vector centro de masa.
    \item $r_cm[2]$: es el vector $\vect{r}_cm$, visto como un arreglo de una matriz de una fila y dos columnas.
    \item $x_max, y_max y x_min, y_min $: Son las componentes del vector posición en los extremos que nos ayudaran a determinar el area del rectangulo que continene las particulas.
    \item $m_t, I y A$: Son la masa total, el momento de inercia y el area del rectangulo respectivamente.
\end{itemize}


\section{Soluciones}

\subsection{Posiciones}

El problema nos da la forma de calcular las componentes del vector posición, esto es las ecuaciones \ref{x_n} y \ref{y_n}. Como se dijo en la sección \ref{analisis}, se busca hacer todos
los calculos necesarios en un solo bucle. 
\begin{equation}
    \label{x_n}
    x_n = -2.7 x_{n-1} \ln(x_{n-1})
\end{equation}

\begin{equation}
    \label{y_n}
    y_n = 0.9 \sin(\pi y_{n-1})
\end{equation}

Por lo tanto, afuera del bucle instanciamos las posiciones iniciales. Dentro del bucle calculamos la posición de cada partícula a partir de la 1 hasta la particula $N-1$, como vamos a hacer todos los calculos necesarios antes de que acabe la iteracion entonces no es conveniente guardar las posciones de las particulas. Por esta razon al final del bucle se le asigna a $x_ant$ el valor de $x_n$ para asi darle paso a la siguente particula y "olvidar la posición de la partícula anterior.

\begin{lstlisting}[language=C, caption={Calculo de componentes de posición}, style=mystyle]
    void main(){
        x_ant = x_0;
        y_ant = y_0;
                
        for (int i=1; i<N; i++){
            // componente del vector posicion en x
            x_n = -2.7*x_ant*log(x_ant);

            // componente del vector posicion en y
            y_n = 0.9*sin(M_PI *y_ant);        
             
            //Pasar a la siguiente particula
            x_ant = x_n;
            y_ant = y_n;
        
            }       
            
            }

    \end{lstlisting}


\subsection{Calculo de masa}

Para el cálculo de la masa se usó la ecuación \ref{masa}, donde nos damos cuenta que se necesita el módulo del vector posición.

\begin{equation}
    \label{masa}
    m_n = |x_n y_n|^{1.3} e^{-r_n / \sqrt{2}}
\end{equation}

Esta ecuacion se usó dentro del mismo bucle del cálculo de las posiciones. No podemos inicializar la masa total en cero debido a que tenemos una particula inicial y la masa inicial del sistema sera la masa de esa partícula.
Entonces, necesitamos calcular afuera del bucle el módulo del vector porición inicial $r_0$ y la masa inicial $m_t = |x_0 y_0|^{1.3} e^{-r_0 / \sqrt{2}}$. Dentro del bucle se hace el cálculo de la masa de cada particula y se hace usp de la recursividad para que $m_t$ sea la suma de todas las masas de las particulas. Esta recurisividad se expresa de la forma $m_t += m_n$, que es lo mismo que decir $m_t = m_t + m_n$. En cristiano, se lee que la masa total del sistema es igual a la masa total del sistema anterior mas la masa de la particula "actual".
\begin{lstlisting}[language=C, caption={Calculo de la masa.}, style=mystyle]
    void main(){
    x_ant = x_0;
    y_ant = y_0;
    double r_0 = sqrt(x_0*x_0 + y_0*y_0);
    m_t = pow(fabs(x_0*y_0), 1.3)*exp(-r_0/sqrt(2));
    
    for (int i=1; i<N; i++){
        // componente del vector posicion en x
        x_n = -2.7*x_ant*log(x_ant);
        // componente del vector posicion en y
        y_n = 0.9*sin(M_PI *y_ant);
        //Modulo del vector posicion
        r_n = sqrt(x_n*x_n + y_n*y_n);
        // Calcular la masa
        m_n = pow(fabs(x_n*y_n), 1.3)*exp(-r_n/sqrt(2));
        m_t += m_n;
        
        //Pasar a la siguiente particula
        x_ant = x_n;
        y_ant = y_n;

    }
    
    printf("La masa total del sistema es: %lf\n", m_t);   
    }

    \end{lstlisting}

\subsection{Calculo del Centro de Masa.}

\begin{equation}
    \label{rcm}
    \Vec{r}_{cm} = \frac{1}{M} \sum_i m_i \vec{r}_i
\end{equation}

\subsection{Calculo de Momento de Inercia}

\begin{equation}
    \label{Inercia}
    I =  \sum_i m_i r^{2}_{i}
\end{equation}

\subsection{Calculo del área mínima del rectángulo que contiene a las partículas.}

 
\end{document}
